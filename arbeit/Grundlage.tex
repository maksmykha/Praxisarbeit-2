% !TEX root =  master.tex
\chapter{Grundlage}
Für eine nachvollziehbare Darstellung des Entwicklungsprozesses ist es erforderlich, die zugrunde liegenden Begriffe und Konzepte präzise und eindeutig zu definieren. Der folgende Abschnitt führt daher zentrale Fachtermini aus den Bereichen der PLM‑Systeme, der Materialwissenschaft sowie der Künstlichen Intelligenz (KI) systematisch ein. Ziel ist es, eine einheitliche begriffliche Grundlage zu schaffen, die sowohl die theoretische Einordnung der Anwendung unterstützt als auch die methodische Absicherung der anschließenden Analyse gewährleistet.
\section{Einführung in das Product Lifecycle Management}
Product Lifecycle Management (PLM) bezeichnet die unternehmerische Tätigkeit, Produkte eines Unternehmens über ihren gesamten Lebenszyklus hinweg effizient zu steuern und zu verwalten – von der ersten Produktidee über Entwicklung, Markteinführung und Nutzung bis hin zur Außerbetriebnahme und Entsorgung. Dabei umfasst PLM sowohl die Verwaltung einzelner Produkte als auch des gesamten Produktportfolios, also der Gesamtheit aller im Unternehmen angebotenen Produkte.
Der Fokus liegt auf der Begleitung der Produkte von der frühen Entwicklungsphase über Wachstums- und Reifephasen bis zum Ende ihres Lebenszyklus. Ziel des PLM ist es, Produktumsätze zu steigern, produktbezogene Kosten zu reduzieren, den Wert des Produktportfolios zu maximieren und sowohl für Kunden als auch für Anteilseigner einen nachhaltigen Mehrwert aus bestehenden und zukünftigen Produkten zu schaffen. \cite{stark2022product}
In 
\subsection{Grundlage zu PLM-Systeme}

\subsection{3DExpirience-Plattform}

\subsection{Künstliche Intelligenz für PLM}

\section{Grundlage zu den KI-Technologien}
\subsection{Einführung in die Machine Learning}
\section{Materialentwicklung in der Automobilindustrie}
\subsection{Machine Learning In der Materialentwicklung}
\subsection{Nachhaltigkeit der Materialien}
